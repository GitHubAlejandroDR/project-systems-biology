\section{Conclusiones}
Tras una comparación de las redes genéticas de Tiroiditis y Hashimoto, finalmente hemos obtenido varios genes candidatos para la expansión de la red de Hashimoto.
El análisis bioinformático realizado ha consistido en varios pasos.El primero fue una expansión de ambas redes hasta obtener 200 genes en total. El siguiente paso fue la agrupación de estos genes en comunidades para su posterior análisis funcional.
Una vez las comunidades estaban asociadas a un pathway biológico, el paso final fue comparar (para ambos fenotipos) que pathways comunes compartían y, a partir de las comunidades asociadas a estos, poder determinar genes pertenecientes a la red de Tiroiditis que pudieran formar parte de la de Hashimoto.

Con estos genes candidatos se pueden realizar análisis posteriores y una investigación biológica más exhaustiva para poder decidir si realmente pueden formar parte de la red de Hashimoto.

A parte de determinar varios genes candidatos  hemos obtenido otros resultados interesantes. Por ejemplo, una relación de la tiroiditis con la enfermedad de Alzheimer. 




