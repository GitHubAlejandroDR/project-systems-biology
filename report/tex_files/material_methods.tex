\section{Materiales y métodos}
\subsection*{Datos fenotipos}
El conjunto inicial de interacciones gen-gen asociados a los fenotipos HP:0100646 Thyroiditis y HP:0000872 Hashimoto fue obtenido a partir de la base de datos STRING (https://string-db.org/). La base de datos STRING  integra conocimiento sobre las relaciones entre proteínas, incluyendo desde interacciones físicas hasta relaciones funcionales. Las relaciones son determinas a partir de una serie de evidencias con una significancia asociada. \cite{Szklarczyk2021}  Las evidencias que determinaron las redes de interacción de nuestros fenotipos son las siguientes: (1) textmining, (2) experimentos, (3) bases de datos, (4) co-expresión, (5) vecindad, (6) fusión de genes, (7) co-ocurrencia. La significancia mínima asociada a cada interacción determinada fue de 0.7. Encontramos un total de 46 genes asociados al fenotipo  HP: 0100646 Thyroiditis y 17 a  HP:0000872 Hashimoto. 
\subsection*{Propagación de red}
Realizamos una propagación de red al conjunto de genes incial asociado a cada fenotipo. Añadimos un total de 200 genes a los conjuntos de genes iniciales.
\subsection*{Enriquecimiento funcional}
El enriquecimiento funcional es crucial en la interpretación de datos ómicos. Realizamos un enriquecimiento funcional KEGG a las comunidades presentes en las redes asociadas a cada fenotipo. Asignamos a cada comunidad la función molecular correspondiente en base al gene ratio obtenido en el enriquecimiento. Fijamos el umbral del p-value en 0.05. \\ 
La Kyoto Enciclopedia de Genes y Genomas (KEGG) es una base de datos aplicada al entendimiento de la información funcional en organismos a partir de su información genética. La base de datos KEGG fué desarollada por los laboratorios Kanehisa en 1995 y actualmente es un referente en la integración y interpretación de datos moleculares generados por secuenciación  genómica u otras tecnologías de alto rendimiento. \\ 
La información se encuentra agrupada en cuatro bloques según su sentido biológico: genes y proteínas (información genómica), sustancias químicas (información química), relaciones de redes (información de sistema), enfermedad y drogas asociadas (información sanitaria). El mayor componente de información se denomina PATHWAY y consiste en diagramas asociados a rutas moleculares, incluyendo rutas metabólicas y regulativas. \\
La información en KEGG esta representada en forma de grafo y técnicas computacionales son aplicadas para detectar relaciones entre las características del grafo y funciones celulares anotadas.  \\
Mapeando genes en el genoma y productos en las rutas moleculares, KEGG predice interacciones entre redes y funciones celulares asociadas. Los genes del genoma son representados como nodos conectados en una dimensión y las rutas moleculares es un grafo de productos de lo genes con un patrón de conexiones mas complejo. \cite{Ogata1999, Minoru}\\
Para el acceso y mapeo de los genes de nuestro experimento en KEGG, utilizaremos el paquete R clusterProfiler.Cluster profiler fue publicado en 2012 y diseñado para realizar análisis de sobrerepresentación usando términos GO y KEGG. Actualmente soporta varias ontologías y anotaciones de pathways, y tiene miles de especies con capacidad de anotación. A través de una pequeña interfaz permite la manipulación y visualización de los resultados de enriquecimientos. Complementando las funcionalidades de clusterProfiler se han incorporado multitud de paquetes. Entre estos GoSemSim para eliminar términos GO redundantes, o enrichPlot para visualizar los resultados del enriquecimiento \cite{Wu2021}.
\subsection*{Búsuqueda de funciones comunes}
\subsection*{Análisis de comunidades}