\section{Discusión}
A raíz de los resultados obtenidos en la comparación de clusters, se ha determinado que hay 3 pathways donde clusters de ambos fenotipos participan: 'JAK-STAT signaling pathway', 'Alzheimer disease' y 'Bacterial invasion of epithelial cells'.Los clusters asociados a cada fenotipo en los pathways 'Bacterial invasion of epithelial cells' y 'Alzheimer disease' son los mismos. No osbtante, el gene ratio asociado al pathway 'Alzheimer disease' (0.1347), es mayor que el asociado a 'Bacterial invasion of epithelial cells' (0.1088). Por esto último descartaremos para el análisis el pathway 'Bacterial invasion of epithelial cells'.
\\ \\
Para el primer pathway los genes 'IL2RG' y 'JAK3' están presentes en el cluster de Hashimoto, así como en el de Tiroiditis. No obstante, unicamente "IL2RG" pertenece al conjunto de genes semilla. Para el segundo pathway solo encontramos un gen que esté en ambos fenotipos, el 'PIK3R2', el cual pertenece a los genes obtenidos a partir de la propagación de red. Se añade que, además, hay indicidios de  interacción física entre los genes 'IL2RG', 'JAK3' y 'PIK3R2' (R-HSA-1295544). Además dichos genes están asociados a la producción de IL-4, citoquina involucrada en procesos de regulación inmunitaria\cite{Gadani2012}. 
\\ \\
Uno de los objetivos del proyecto, y a modo de líneas futuras, era intentar expandir la red de genes de Hashimoto a partir de genes pertenecientes a la de Tiroiditis. Anteriormente hemos explicado que existen 3 genes que se encuentran en clusters de ambos fenotipos. Pero para este caso, nos interesan aquellos que están presentes en Tiroiditis pero no en Hashimoto. Unos posibles candidatos serían, basándonos en el pathway de 'vía de señalización JAK-STAT' , los genes 'IL2RB', 'EPOR','IL10RA', 'PDGFRA','PDGFRB','IL12RB2','IL27RA','IFNAR1','IL22RA1',
'EGFR','IL10RB','IFNGR2' y 'IL7R'. Basándonos en el segundo patwhay 'Alzheimer disease', encontramos los siguientes genes candidatos: 'PI4KA', 'PIK3R3' y 'PIK3R1'.
\\ \\
Recalcar que todos estos genes son candidatos y sería necesario un análisis posterior para poder determinar más relaciones de estos genes con los ya presentes en la red de genes de Hashimoto.

